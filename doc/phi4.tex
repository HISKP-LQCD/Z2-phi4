\documentclass[10pt,a4paper]{article}
 \usepackage[utf8]{inputenc}
 \usepackage[english]{babel}
 \usepackage{amsmath}
 \usepackage{amsfonts}
 \usepackage{amssymb}
 \usepackage{graphicx}
 \usepackage{tcolorbox}
 \usepackage[left=2cm,right=2cm,top=2cm,bottom=2cm]{geometry}
 \begin{document}
 
 \section{Lagrangian}
 \begin{align}
 & {\cal L}= {\cal L}_0+ {\cal L}_1+  {\cal L}_{IZ_2} \\
 &{\cal L}_0= \frac{1}{2} \partial_\mu \varphi_0\partial_\mu \varphi_0 +\frac{1}{2}m_0 \varphi_0^2 +\lambda_0 \varphi_0^4 \\
 &{\cal L}_1=\frac{1}{2} \partial_\mu \varphi_1\partial_\mu \varphi_1 +\frac{1}{2}m_1 \varphi_1^2+\lambda_1 \varphi_1^4 \\
 &{\cal L}_{IZ_2}= \mu \varphi_0^2 \varphi_1^2 \\
  &{\cal L}_{I}= g \varphi_1 \varphi_0^3
 \end{align}
On the lattice we can discretise the derivative $\partial_\mu \varphi(x)=\frac{1}{a}(\varphi(x+\mu)-\varphi(x)) $.
On the lattice the above lagrangian can be written in a more convenient way for simulations
\begin{align}
 &{\cal L}_0= \sum_{x}\left[ -2\kappa_0\sum_\mu \phi_0(x) \phi_0({x+\mu}) +\lambda_L^0 \left( \phi_0(x)^2-1\right)^2+ \phi_0(x)^2  \right]\,,\\
 &{\cal L}_1= \sum_{x}\left[ -2\kappa_1\sum_\mu \phi_1(x) \phi_1({x+\mu}) +\lambda_L^1 \left( \phi_1(x)^2-1\right)^2+ \phi_1(x)^2  \right]\,,\\
 &{\cal L}_{IZ_2}= \mu_L  \sum_{x} \phi_0(x)^2 \phi_1(x)^2 \,,\\
  &{\cal L}_{I}= g_L \sum_{x} \phi_1(x) \phi_0(x)^3 \,.
 \end{align}
With
\begin{align}
& m_0^2=\frac{1-2\lambda_L^0}{\kappa_0}-8\,,\quad \lambda_0=\frac{\lambda_L^0}{4\kappa_0^2}\,,\quad \varphi_0=\sqrt{2\kappa_0}\phi_0\\
& m_1^2=\frac{1-2\lambda_L^1}{\kappa_1}-8\,,\quad \lambda_1=\frac{\lambda_L^1}{4\kappa_1^2}
\,,\quad \varphi_1=\sqrt{2\kappa_1}\phi_1 \,,
 \end{align}
 and
 \begin{align}
\mu=\frac{\mu_L}{4\kappa_0\kappa_1} \,,\quad g=\frac{g_L}{4\sqrt{\kappa_0}\kappa_1^{3/2}}
 \end{align}
 
\section{BH four point function}
 


$$C_4^{BH}=\frac{\langle \phi_0(\frac{T}{2})\phi_1(t)\phi_1(\frac{T}{8}) \phi_0(0)\rangle}
{\langle \phi_0(\frac{T}{2}) \phi_0(0)\rangle \langle \phi_1(t)\phi_1(\frac{T}{8}) \rangle} -1
$$

\subsection{Spectral Decomposition for $t_1<t_2<t_3<t_4$}

\begin{itemize}

\item Numerator
\begin{gather}
\langle \phi_0(t_1)\phi_1(t_2)\phi_1(t_3) \phi_0(t_4)\rangle=\\
 \sum_{i,j,k}\frac{1}{ 2 m_j 2 m_k}e^{-(T-t_1)E_i} e^{-t_1E_j}\langle i| \phi_0 | j\rangle\langle j|  \phi_1(t_2)\phi_1(t_3) 
 | k\rangle\langle k|  \phi_0 |i\rangle e^{-(E_i-E_k)t_4}\\
\end{gather}
in the limit $T-t_1\to \infty$ and setting $t_4=0$
\begin{gather}
 \sum_{j,k}\frac{1}{ 2 m_j 2 m_k}\langle 0| \phi_0 | j\rangle  e^{-t_1E_j} \langle j|   \phi_1(t_2)\phi_1(t_3) 
 | k\rangle  \langle k| \phi_0 | 0\rangle  \\
 = \sum_{j,k}\frac{1}{ 2 m_j 2 m_k}\langle 0| \phi_0 | j\rangle  e^{-(t_1-t_2)E_j} \langle j|   \phi_1 e^{-(t_2-t_3)H}\phi_1 
 | k\rangle  e^{- t_3 E_k} \langle k| \phi_0 | 0\rangle
\end{gather}
Assuming $t_1-t_2>>0$ and $t_3>>0$ the states $|j\rangle =| \pi\rangle$ and 
$|k\rangle =| \pi\rangle$ so we get
\begin{gather}
=\frac{1}{ 2 m_\pi 2 m_\pi}|\langle 0| \phi_0 | \pi\rangle|^2  e^{-(t_1-t_2)E_{\pi}}e^{- t_3 E_\pi}  \langle \pi|   \phi_1 e^{-(t_2-t_3)H}\phi_1 
 | \pi\rangle  
\end{gather}
Setting  $t_1=T/2$, $t_2=t$ and $t_3=T/8$
\begin{gather}
\langle \phi_0(\frac{T}{2})\phi_1(t)\phi_1(\frac{T}{8}) \phi_0(0)\rangle=
\frac{1}{ 2 m_\pi 2 m_\pi}|\langle 0| \phi_0 | \pi\rangle|^2  e^{-\frac{T}{2}E_{\pi}}e^{- (\frac{T}{8}-t) E_\pi}  \langle \pi|   \phi_1 e^{-(t-\frac{T}{8})H}\phi_1 
 | \pi\rangle  
\end{gather}

 \item Denominator 0
$$
\langle \phi_0(\frac{T}{2}) \phi_0(0)\rangle = \frac{1}{m_\pi}|\langle 0| \phi_0 | \pi\rangle|^2  e^{-\frac{T}{2}E_\pi}
$$
Where we have summed both the forward and backward signal

 \item Denominator 1
$$
\langle \phi_1(t) \phi_1(\frac{T}{8})\rangle = \frac{1}{2m_N}|\langle 0| N | \pi\rangle|^2  \left(e^{- (t-\frac{T}{8})E_N}+  e^{(T-t+\frac{T}{8})E_N}  \right)
=\frac{1}{2m_N}|\langle 0| N | \pi\rangle|^2  e^{- (t-\frac{T}{8})E_N}
$$
We can ignore the second term since $T/8<t<T/2$

\end{itemize}

Putting the various pieces toghether we get
\begin{gather}
C_4^{BH}=\frac{m_N}{ 2 m_\pi }\frac{
  e^{ (t-\frac{T}{8})(E_N+E\pi)} \langle \pi|   \phi_1 e^{-(t-\frac{T}{8})H}\phi_1 
 | \pi\rangle  
}{
 |\langle 0| N | \pi\rangle|^2  
}-1
\end{gather}
\begin{tcolorbox}
{To be compared with the expression on the  paper BH:
 \begin{multline}
\label{eq:cThetaNpiDef}
c^{\Theta, N \pi}_{\vec q_1' \vec q_2' \vec q_1 \vec q_2}(t', t \vert \, M_0) \equiv \frac{ 2 \omega_{\vec q_2'} e^{\omega_{\vec q_2'} t' } 2 \omega_{\vec q_2 } e^{- \omega_{\vec q_2 } t } }{Z_N} C_{a'b'} C_{ab} \, \\
\times \langle \pi^{a'} , \vec q_1' \vert \widetilde N^{b'}_{ \vec q_2'}(0) \, \Theta(\hat M - M_0, \Delta) \, e^{- \hat H(t' - t)} e^{ \omega_{\vec q_1'} t'} e^{ - \omega_{\vec q_1} t} \, \widetilde N^{\dagger b}_{-\vec q_2}(0) \vert \pi^{a}, \vec q_1 \rangle \,.
\nonumber
\end{multline}
}
\end{tcolorbox}

\begin{tcolorbox}
where $\omega_{\vec q} = \sqrt{\vec q^2 + m_\pi^2}$ and $Z_\pi = \langle \pi, \vec p \vert \pi(0) \vert 0 \rangle^2$. Here the single particle state has the usual relativistic normalization, $\langle \pi, \vec q \vert \pi, \vec q' \rangle = 2 \omega_{\vec q} (2 \pi)^3 \delta^3(\vec q - \vec q')$.
\end{tcolorbox}
Comparing the two equation above we get 
\begin{equation}
8m_\pi m_N C_4^{BH}=  c^{\Theta, N \pi}_{\vec q_1' \vec q_2' \vec q_1 \vec q_2}(t', t \vert \, M_0)
\end{equation}
 
 \begin{tcolorbox}
 To be compared with the expression on the  paper BH:
 \begin{multline}
c^{\Theta, N \pi}_{\vec 0}(t', t \vert \, M_0) _{\sf c} = 8 \pi (m_N + m_\pi) \, a_{N\pi} \, (t' - t ) \\
- 16 \, a_{N \pi}^2 \sqrt{2 \pi (m_N + m_\pi) m_N m_\pi (t'-t)} + \mathcal O \big ((t'-t)^0\big) \,.\nonumber
\end{multline}
 \end{tcolorbox}
Finally our formula become 

\begin{multline}
 C_4^{BH}=  \frac{1}{8m_\pi m_N}\left[ 8 \pi (m_N + m_\pi) \, a_{N\pi} \, (t' - t ) 
- 16 \, a_{N \pi}^2 \sqrt{2 \pi (m_N + m_\pi) m_N m_\pi (t'-t)} + \mathcal O \big ((t'-t)^0\big) \right]\,.\nonumber
\end{multline} 
 
 
 
 
 \end{document} 
